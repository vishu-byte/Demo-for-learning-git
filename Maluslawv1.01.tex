\documentclass[11pt,a4paper]{article}
\usepackage{newtxtext,newtxmath}
\usepackage{fancyhdr}
\usepackage{hyperref}
\usepackage{amsmath,array,graphicx}
\usepackage{kantlipsum}
\usepackage{graphicx}
\usepackage{xcolor}
\graphicspath{ {./images/} }
\usepackage{booktabs}
\usepackage{multirow}
 \usepackage{booktabs}



\fancyhf{ }

\rhead{\thepage}
\pagestyle{fancy}

\begin{document}
%%%% HEADING %%%%

\lhead{Kirori Mal College, University of Delhi}


%Title
\title{\textbf{To verify git}}
\author{Vishu Saini\\
	Course:  Bsc.(H.)Physics-III Year\\
	Roll Number:  31830157\\			
	University Roll Number: 18036567119\\
	Instructor: Dr. Sangeeta Gadre}
\date{9 June, 2021}
\maketitle
%
%\begin{table*}[thb]
%\centering\small
%\begin{tabular}{l r}
%Course: & Bsc.(H.)Physics-III Year\\
%Roll Number: & 31830157\\
%University Roll Number: &18036567119\\
%Instructor:& Dr. Sangeeta Gadre\\
%\end{tabular}


%
%%abstract
%\begin{abstract}
%In this study, we shall investigate how current-voltage characteristics of a p-n junction diode can be used to determine the ideality factor $\eta$ for the diode. We shall use previously determined value of Boltzmann consant, k=1.38x$10^{-23}$J$K^{-1}$(recommended by CODATA 2018) 
%\end{abstract}

\thispagestyle{fancy}

%%%% BODY %%%%

\section*{AIM}
To verify the Law of Malus for a plane polarized light.

\section*{APPARATUS}
A lamp, a convex lens, an optical bench, a polarizer and an anaylser with graduated circular scales, a photo-cell with a digital voltmeter or a millammeter.

\section*{THEORY AND FORMULAE USED}
\textit{Law of Malus} tells us how the intensity transmitted by the analyser varies with the angle that its plane of transmission makes with that of the polarizer.
It states that when a beam of completely plane polarized light is incident on an analyser, the resultant intensity of light $I$ transmitted from the analyser varies directly as the square of the cosine of the angle
$\phi$ between the plane of transmission of analyser and polarizer. ie.

\begin{equation}
I \varpropto \cos^{2}\phi \implies I = I_{0} \; \cos^{2}\phi
\end{equation}

where, $I_{0} $ =  intensity of the plane polarized light incident on the analyser.



\section*{OBSERVATIONS}
\begin{tabular}{l r}
Least Count of Voltmeter, $\Delta V = 0.001\; V $ \\
Least Count of Circular Scale of polarizer and analyser = $ 0.1^{\circ} $ \\
Angle of polarizer, $\theta_{p} = 90.0^{\circ} $
\end{tabular}

(Refer to Table 1 for Observations)

% Please add the following required packages to your document preamble:

\begin{table}[]
\centering
\begin{tabular}{@{}lll@{}}
\toprule
\multicolumn{1}{c}{S No.} & \multicolumn{1}{c}{\begin{tabular}[c]{@{}c@{}}Angle of analyser\\ (Degrees)\end{tabular}} & \multicolumn{1}{c}{\begin{tabular}[c]{@{}c@{}}Intensity\\ (V)\end{tabular}} \\ \midrule
1                         & 0                                                                                         & -2.42                                                                       \\
2                         & 10                                                                                        & 0.267                                                                       \\
3                         & 20                                                                                        & 0.308                                                                       \\
4                         & 30                                                                                        & 0.311                                                                       \\
5                         & 40                                                                                        & 0.347                                                                       \\
6                         & 50                                                                                        & 0.358                                                                       \\
7                         & 60                                                                                        & 0.365                                                                       \\
8                         & 70                                                                                        & 0.371                                                                       \\
9                         & 80                                                                                        & 0.375                                                                       \\
10                        & 90                                                                                        & 0.376                                                                       \\
11                        & 100                                                                                       & 0.375                                                                       \\
12                        & 110                                                                                       & 0.373                                                                       \\
13                        & 120                                                                                       & 0.368                                                                       \\
14                        & 130                                                                                       & 0.362                                                                       \\
15                        & 140                                                                                       & 0.351                                                                       \\
16                        & 150                                                                                       & 0.339                                                                       \\
17                        & 160                                                                                       & 0.318                                                                       \\
18                        & 170                                                                                       & 0.288                                                                       \\ \bottomrule
\end{tabular}
\end{table}


% Please add the following required packages to your document preamble:

\begin{table}[]
\centering
\begin{tabular}{@{}lll@{}}
\toprule
\multicolumn{1}{c}{S No.} & \multicolumn{1}{c}{\begin{tabular}[c]{@{}c@{}}Angle of analyser\\ (Degrees)\end{tabular}} & \multicolumn{1}{c}{\begin{tabular}[c]{@{}c@{}}Intensity\\ (V)\end{tabular}} \\ \midrule
19                        & 180                                                                                       & -2.032                                                                      \\
20                        & 190                                                                                       & 0.262                                                                       \\
21                        & 200                                                                                       & 0.307                                                                       \\
22                        & 210                                                                                       & 0.330                                                                       \\
23                        & 220                                                                                       & 0.345                                                                       \\
24                        & 230                                                                                       & 0.355                                                                       \\
25                        & 240                                                                                       & 0.364                                                                       \\
26                        & 250                                                                                       & 0.370                                                                       \\
27                        & 260                                                                                       & 0.374                                                                       \\
28                        & 270                                                                                       & 0.375                                                                       \\
29                        & 280                                                                                       & 0.374                                                                       \\
30                        & 290                                                                                       & 0.372                                                                       \\
31                        & 300                                                                                       & 0.368                                                                       \\
32                        & 310                                                                                       & 0.360                                                                       \\
33                        & 320                                                                                       & 0.351                                                                       \\
34                        & 330                                                                                       & 0.337                                                                       \\
35                        & 340                                                                                       & 0.318                                                                       \\
36                        & 350                                                                                       & 0.285                                                                       \\
37                        & 360                                                                                       & -2.597                                                                      \\ \bottomrule
\end{tabular}
\end{table}


\section*{CACULATIONS}
\begin{enumerate}
\item The graph of $I \; vs \; \theta_{a}$ is plotted using python. (Figure 1) As observed, peak occurs at $\theta_{a} = 90^{\circ}$
\item The graph of $I \;vs\; \cos^{2}\phi$ is also plotted using python. (Figure 2)  ($\phi = \theta_{a} - 90^{\circ}$)
\end{enumerate}

Here, $\phi $ =  angle between polarizer and analyser axes.( Python Code is provided in Appendix)






\begin{figure}[thb]
\centering
\includegraphics[width=6cm,height=6cm]{plot1}

\caption{Plotting the graphs using python}
\end{figure}


\begin{figure}[thb]
\centering
\includegraphics[width=6cm,height=6cm]{plot2}

\caption{Plotting the graphs using python}
\end{figure}









\section*{RESULT}
\begin{enumerate}
\item The graph of $I \; vs \; \theta_{a}$ shows that intensity is maximum for two positions( $90^{\circ} \; and\; 270^{\circ}$) of the analyser and minimum in between.
\item The graph of $I \;vs\; \cos^{2}\phi$ is approximately a straight line thus verifying the Malus' law.  
\end{enumerate}



\end{document}





