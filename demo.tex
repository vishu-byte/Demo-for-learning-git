\documentclass[11pt,a4paper]{article}
\usepackage{newtxtext,newtxmath}
\usepackage{fancyhdr}
\usepackage{hyperref}
\usepackage{amsmath,array,graphicx}
\usepackage{kantlipsum}
\usepackage{graphicx}
\usepackage{xcolor}
\graphicspath{ {./images/} }
\usepackage{booktabs}
\usepackage{multirow}
 \usepackage{booktabs}



\fancyhf{ }

\rhead{\thepage}
\pagestyle{fancy}

\begin{document}
%%%% HEADING %%%%

\lhead{Kirori Mal College, University of Delhi}


%Title
\title{\textbf{To verify git}}
\author{Vishu Saini\\
	Course:  Bsc.(H.)Physics-III Year\\
	Roll Number:  31830157\\			
	University Roll Number: 18036567119\\
	Instructor: Dr. Sangeeta Gadre}
\date{9 June, 2021}
\maketitle
%
%\begin{table*}[thb]
%\centering\small
%\begin{tabular}{l r}
%Course: & Bsc.(H.)Physics-III Year\\
%Roll Number: & 31830157\\
%University Roll Number: &18036567119\\
%Instructor:& Dr. Sangeeta Gadre\\
%\end{tabular}


%
%%abstract
%\begin{abstract}
%In this study, we shall investigate how current-voltage characteristics of a p-n junction diode can be used to determine the ideality factor $\eta$ for the diode. We shall use previously determined value of Boltzmann consant, k=1.38x$10^{-23}$J$K^{-1}$(recommended by CODATA 2018) 
%\end{abstract}

\thispagestyle{fancy}

%%%% BODY %%%%





\section*{RESULT}
\begin{enumerate}
\item The graph of $I \; vs \; \theta_{a}$ shows that intensity is maximum for two positions( $90^{\circ} \; and\; 270^{\circ}$) of the analyser and minimum in between.
\item The graph of $I \;vs\; \cos^{2}\phi$ is approximately a straight line thus verifying the Malus' law.  
\end{enumerate}



\end{document}





